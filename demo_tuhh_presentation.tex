\documentclass[english,institute=none]{tuhh_presentation}

\usepackage[utf8]{inputenc}
\usepackage[T1]{fontenc}

\title{Title/Headline of Hamburg University of Technology Presentation}
\date{\today} % or \date{dd}{mm}{yyyy} or \date{Text}
\author[Corresponding Author]{First Author, Second Author}
\email{corresponding.author@tuhh.de}
%\institute{Corresponding Author's Institute} % Optional: Please refer to https://collaborating.tuhh.de/e-4/tuhh_latex_presentation/-/wikis/Slides-with-Institute-Logos .
\street{Am Schwarzenberg-Campus 1}
\city{21073 Hamburg}
\website{www.tuhh.de}
\telephonenumber{+49 40 42878-3330}

%\autofontdecrement % This commands will automatically reduce font size for itemize/enumerate environments

\addbibresource{demo_tuhh_presentation.bib}

\begin{document}

\titlepage
\agenda

\section{Environments}\label{se:environments}
\begin{frame}{List Environments}
    \begin{minipage}{0.49\textwidth}
        \begin{itemize}
            \item This is an item.
            \item This is a second item.
            \begin{itemize}
                \item This is a sub item.
                \item This is a second sub item.
            \end{itemize}
            \item This is a third item.
        \end{itemize}
    \end{minipage}
    \begin{minipage}{0.49\textwidth}
        \begin{enumerate}
            \item This is an item in an enumeration.
            \item This is a second item in an enumeration.
            \item This is a third item in an enumeration.
        \end{enumerate}
    \end{minipage}
\end{frame}

\begin{frame}{Block Environments}
    \begin{block}{My Block}
      This is a block with a title.
    \end{block}

    \begin{definition}
      A \emph{sequence} is a mapping from $\mathbb{N}$ to a field of numbers.
    \end{definition}

    \begin{theorem}[Bolzano--Weierstrass]
      Every bounded sequence of real numbers has a convergent subsequence.
    \end{theorem}

    \begin{proof}
      Here comes the proof.
    \end{proof}
\end{frame}

\begin{frame}{Figure Environment}
    \begin{figure}
        \centering
        \includegraphics[width=0.8\textwidth]{tuhh_theme/imgs/TUHH_building-a.pdf}
        \caption{Figures work as usual and support JPG, PNG or PDF files.\label{fig:test_image}}
    \end{figure}
\end{frame}

\section{Fonts}
\begin{frame}{Font Sizes}
    \centering{\textbf{This is the place where you can write your content in bold.}}

    \centering{\textit{This is the place where you can write your content in italic.}}

    \centering{\textbf{\textit{This is the place where you can write your content in bold and italic.}}}

    \centering{\LARGE This is the place where content is LARGE.}

    \centering{But if your sentences get too long, you cannot be sure that the whole sentence will fit into the same line - You might get a linebreak!}
\end{frame}

\begin{frame}{Footnotes}
    You can use footnotes\footnote{Hello from down here!}!

    You can even have multiple footnotes\footnote{Hello again!}!

    What about really long footnotes\footnote{Let's see how long we can make this footnote, just to make sure you can put everything you want here!}?
\end{frame}


\section{Tables}
\begin{frame}{Tables}
    \begin{table}
        \begin{tuhhtabular}{|l|l|}
            \multicolumn{2}{c}{\tuhhHead{TU Table for Presentations}} \\
            Row 1 & 1 \\
            \hline
            Row 2 & 2 \\
            \hline
            Row 3 & 3 \\
            \hline
            Row 4 & 4 \\
            \hline
            Row 5 & 5 \\
            \hline
            Row 6 & 6 \\
            \hline
        \end{tuhhtabular}\quad\quad
        \begin{tuhhtabular}{|l|l|}
            \tuhhHead{Column 1} & \tuhhHead{Column 2} \\
            Row 1 & 1 \\
            \hline
            Row 2 & 2 \\
            \hline
            Row 3 & 3 \\
            \hline
            Row 4 & 4 \\
            \hline
            Row 5 & 5 \\
            \hline
            Row 6 & 6 \\
            \hline
        \end{tuhhtabular}
    \end{table}
\end{frame}

\section{Miscellaneous}
\begin{frame}{Miscellaneous}
    Collection of \LaTeX~commands and their appearance in this template:
    \begin{itemize}
        \item Citations using \textit{cite}: \cite{tuhh2022demo}
        \item Hyperlinks using \textit{href}: \href{https://www.tuhh.de}{https://www.tuhh.de}
        \item Hyperlinks using \textit{url}: \url{https://www.tuhh.de}
        \item {\small Hyperlinks adjusting to context font size: \url{https://www.tuhh.de}}
        \item Document references using \textit{ref}: Section \ref{se:environments}
    \end{itemize}
\end{frame}

\begin{frame}{Math}
    \begin{theorem}
        For all $p \in \big[\frac{4}{3}, 4\big]$ and all $0 < \theta < 1$, the continuous embedding
        \begin{align*}
            \mathrm{H}^{2 \theta , p}_{0 , \sigma} (\Omega) \subset \mathcal{D}(A_p^{\theta})
        \end{align*}
        holds.
        Furthermore, there exists $\delta \in (0 , 1]$ such that, if $\theta$ and $p$ additionally satisfy either
        \begin{align*}
            \theta < \frac{1}{2} + \frac{1}{2 p} \quad &\text{if} \quad \frac{1}{2} - \frac{1}{p} \leq \frac{\delta}{2} \qquad\text{or } \\
            \theta < \frac{1}{p} + \frac{1 + \delta}{4} \quad &\text{if} \quad \frac{1}{2} - \frac{1}{p} > \frac{\delta}{2},
        \end{align*}
        we have with equivalent norms that $\mathcal{D}(A_p^{\theta}) = \mathrm{H}^{2 \theta , p}_{0 , \sigma} (\Omega).$
    \end{theorem}
\end{frame}

\begin{frame}[notitle]{Title}
    \begin{center}
        You can create frames without the titlebar by the \texttt{notitle} frame type option.
        
        The frame title can be left empty to achieve maximum slide space for large content.
    \end{center}
\end{frame}

\finalpage

\begin{frame}{References}
    \printbibliography
\end{frame}

\end{document}
